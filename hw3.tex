\documentclass[12pt,letterpaper, onecolumn]{exam}
\usepackage{amsmath}
\usepackage{amssymb}
\usepackage{graphicx}
\usepackage{setspace}
\usepackage{nicefrac}
\setcounter{MaxMatrixCols}{20}
\usepackage[lmargin=.75in, rmargin=.75in]{geometry}  %For centering solution box
\lhead{Optimal Estimation}
\rhead{Noah Miller}
\thispagestyle{empty}   %For removing header/footer from page 1

\begin{document}

\begingroup
\centering
\LARGE Optimal Estimation\\
\LARGE Homework 3 \\[0.5em]
\large \today\\[0.5em]
\large Noah Miller\par
\large 903949330\par
\large MECH 7710\par
\endgroup
\pointsdroppedatright   %Self-explanatory
\printanswers
\renewcommand{\solution}{\noindent\textbf{Ans:}\enspace}   %Replace "Ans:" with starting keyword in solution box
\vspace{.5cm}

\noindent Note: All problems are to be samples at 10~Hz ($T_s = 0.1$) except for problem 3.

\noindent Also, all data is available on Canvas.
\begin{questions}
    \question{Kalman Filter at its best -- simulation (actually the Kalman filter is also quite reliable when we have an excellent model and low noise sensors). Suppose we have a second-order system that we are regulating about zero (position and velocity) by wrapping an \textit{optimal} control loop around the system. The new dynamics of the continuous time system are given by the closed-loop A matrix:
        \[
            A_{CL} =
            \begin{bmatrix}
                0  & 1    \\
                -1 & -1.4 \\
            \end{bmatrix}
        \]
        Suppose our measurement is simply position $\left(C = \left[\;1 \quad 0\;\right]\right)$. There is a white noise process disturbance $\left(force, B_w = \left[\;0 \quad 1\;\right]^T\right)$ acting on the controlled system.}
    \begin{parts}
        \part{Simulate the controlled system with the disturbance force $\left(1\sigma = 2\right)$ and a sampled sensor noise $\left(1\sigma = 1\right)$ for 100 seconds at a 10~Hz sample rate.}

        \part{What is $Q$,$Q_d$, and $R_d$?}

        \part{Calculate the steady state Kalman gain for the system. This can be done in one of many ways: iterate the Kalman filter unit it converges, $dlqe.m$, $dare.m$, $kalman.m$, $dlqr.m$ ($+$ predictor to current estimator trick), etc.
            What is the steady state covariance of the estimates after the time update, $P^{(-)}$, as well as after the measurement update,$P^{(+)}$? Where are the poles of the estimator?}

        \part{Now use the steady state Kalman filter to generate an estimate $\big(\hat{x}\;\text{and}\;\hat{\dot{x}}\big)$ of the 2 states over time. Calculate the norm of the standard deviation of the errors of each state. \[N = \sqrt{\big(std\big(\dot{x} - \hat{\dot{x}}\big)\big)^2 + \left(std\big(x - \hat{x}\big)\right)^2} \]}

        \part{%
            Change the ratio of the $Q$ and $R$ weights in the Kalman filter design (and repeat part (d) with new Kalman gain but DO NOT regenerate a new $x$ and $\dot{x}$) and determine the effect on the estimation errors. For what ratio of $Q$ and $R_d$ are the errors minimized?

            Note: Often in practice we do not know the actual $Q$ and $R_d$ so these tend to be "tuning" parameters we can use to tune our filter. However, according to Kalman, the estimation errors are only minimized if we use the $Q$ and $R_d$ of the physical system.}

    \end{parts}
    \clearpage
    \question{Download the data \textit{hw3\_2}} from the website. The data is in the form $\left[\;\mathbf{t} \quad \mathbf{y}\;\right]$. Suppose we want to design an estimator to estimate the bias in the measurement $y$. We believe that the bias,$x$, is constant so we use the model given by
    \[\dot{x} = 0\]
    \[y_k = x_k + \nu_k \]
    \[\nu_k \sim N(0,1) \]
    \begin{parts}
        \part{Run the Kalman filter estimator with $Q_d = 0$. What at happens when $t$ becomes greater than 100 seconds? Why? Calculate the steady state Kalman gain $L_{ss}$. Plot $L(k)$. This is known as the filter "going to sleep" (i.e. becomes a least squares estimator). }

        \part{To offset this problem we will "tune" $Q_d$ to track the bias. What is the effect of changing $Q_d$ on the ability to track the step change in the bias? Try values of $Q_d$ from $0.0001$ to $0.01$ and plot $L(k)$ as well as the estimate of the bias $(\hat{x})$. What is the trade-off?}

        \part{Now filter the measurement using the first-order low-pass filter 
        \[H(z)=  \frac{\sqrt{Q_d}}{z - \left(1 - \sqrt{Q_d}\right)} \]
        Use the command: 
        \[yf = filter(numd,dend,y,y0)\]}

        \part{How does this compare to the Kalman filter solution? Why are these two filters the same for this problem.}
    \end{parts}
    \clearpage
    \question{}
    \begin{parts}
        \part{}

        \part{}

        \part{}

    \end{parts}
    \clearpage
    \question{}
    \begin{parts}
        \part{}

        \part{}

        \part{}

        \part{}

    \end{parts}
    \clearpage
    \question{}
    \begin{parts}
        \part{}

        \part{}

        \part{}

        \part{}

        \part{}
    \end{parts}
\end{questions}

\end{document}